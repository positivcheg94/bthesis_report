\section{Висновки до розділу}

У цьому розділі представлені практичні результати реалізованих алгоритмів.Кожен підхід має свої сильні та слабкі сторони, які були описані у порівнянні алгоритмів.

Віднімання фону дало відносно непогані результати в середньому 94\% по першому притерію, та 0.5\% по другому за умов що на відео не були присутні ішні рухомі об'єкти крім руки. Цей алгоритм можна запускати на бюджетних ПК.

Байесовський класифікатор дав найгірші результати - в середньому 80\% по першому критерію та 4\% по другому. Проте його сильними сторонами є те, що класифікація взагалі не потребує ніяких обчислень, а лише доступу до відповідного елемента матриці. Також в опрацьованих джерелах були наведені підходи до пристосування класифікатора до змінних умов освітлення, що у загальному випадку може покращити середні показники критеріїв за стабільних умов.

Робота з камерою глибини показала найкращі показники ... . За умови отримання лише відеопотоку з камери глибини без кольорового відеопотоку цей підхід також не потребує складних обчислень. У випадку, коли потрібні обидва видеопотоки, виникає проблема синхронізації двох відеопотоків і швидкість отримання фреймів збільшується на два порядки. Основним недоліком цього підходу є те, що потрібно використовувати спеціальне обладнання.