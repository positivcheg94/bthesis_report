\section{Обгрунтування вибору платформи та мови програмування}

На початкових етапах виконання роботи основною мовою програмування для реалізації поставлених задач була мова Python. Ця мова була вибрана через простоту та зручність розробки, проте для реалізації байесовського класифікатора вона не підходить оскільки швидкість доступу до елементів матриці np.ndarray дуже низька, а класифікація пікселів побудована саме на цьому.

Таким чином було вирішено зробити аналогічну реалізацію на С++ та в разі великої різниці у часі обробки класифікатором одного фрейма продовжити розробку на С++. 

Реалізація байесовського класифікатора на мові Python в секунду обробляла майже 2 фрейма, в той час як С++ версія коливалася в межах 300-400 фреймів в секунду.