\section{Порівняння алгоритмів}

Усі алгоритми по першому критерію подолали планку 75\%, а в середньому 80\%. По другому критерію усі алгоритми не переходять межі у 5\%. Це значить, що вони не генерують великі помилкові області, які помічаються як ті, що містять людську руку.

\subsection{Віднімання фону}

Сильні сторони:
\begin{enumerate}
	\item Хороша точність - у межах 95\% по першому критерію;
	\item Відсутність складних обчислень - кожна ітерація отримання поточного фрему супроводжується обрахуванням лінійної комбінації заданої кількості зображень;
	\item Відсутня прив'язка до спеціального обладнання.
\end{enumerate}

Слабкі сторони:
\begin{enumerate}
	\item Зі збільшенням кількості рухомих об'єктів на відео алгоритм помічає їх усі, а тому з'являється потреба у більш складному алгоритмі, який буде по формі класифікувати об'єкти;
	\item Чутливий до зміни освітлення - зміна освітлення призведе до невизначеної поведінки протягом повного перепису історії моделі віднімання фону;
	\item Зміна положення камери повністю руйнує логіку алгоритму та призводить до невизначеної поведінки.
\end{enumerate}

\subsection{Байесовський класифікатор}

Сильні сторони:
\begin{enumerate}
	\item Достатня точність у середньому до 80\%;
	\item Відсутність складних обчислень - класифікація пікселя зводиться лише до перевірки ймовірності його кольору бути кольором людської шкіри;
	\item Відсутня прив'язка до спеціального обладнання;
	\item Рух камери у загальному випадку не впливає на класифікацію.
\end{enumerate}

Слабкі сторони:
\begin{enumerate}
	\item Для стійкості до зміни характеру освітлення потрібно тренувати додаткові моделі та на етапі класифікації проводити вибір однієї з них;
	\item Через прив'язку до кольору шкіри помічає усі області, де її знаходить і тому у загальному випадку потрібно проводити їх обробку.
\end{enumerate}

\subsection{Камера глибини}

Сильні сторони:
\begin{enumerate}
	\item Найкраща точність - до 97\% по першому критерію;
	\item У випадках непотрібності кольорового каналу відео відсутність складних обчислень;
	\item Рух камери не впливає на роботу алгоритму взагалі.
\end{enumerate}

Слабкі сторони:
\begin{enumerate}
	\item Потрібна спеціальна камера;
	\item У разі необхідності кольорового каналу значно збільнується кількість обчислень через синхронізацію двох відеопотоків.
\end{enumerate}