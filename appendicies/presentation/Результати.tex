\begin{frame}{Результати роботи}
	\manimate
	\begin{center}
		Віднімання фону у спеціальних умовах дає стабільний результат в середньому 95\% по першому критерію та не більше 1\% по другому.
		
		Байесовський класифікатор із використанням алгоритмів фільтрації ймовірностей досягає в середньому помітки у 80\% по першому, та до 5\% по другому. Низькі показники алгоритм компенсує простотою реалізації та можливістю подальшого удосконалення шляхом пристосування до характеру освітлення.
		
		Використання камери глибини дає найкращий результат по першому критерію, що очікувано, та в середньому 1.5\% по другому через неточності на границі руки.
	\end{center}
\end{frame}