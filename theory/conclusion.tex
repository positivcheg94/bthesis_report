\section{Висновки за розділом}

У цьому розділі розглянуті теоретичні засади колірних просторів, попередньої обробки зображень  та вибраних алгоритмів локалізації людської руки на відео.

Окрім класичного метода віднімання фону розглянуто байесовський класифікатор та запропоновані модифікації моделі. Також через відносну незручність класичного методу навчання класифікатора запропонований метод швидкого навчання за допомоги камери Intel Realsense F200. Останнім було розглянуто більш сучасний метод локалізації руки на відео, що потребує камеру від компанії Intel, як приклад сучасного підходу до розв'язання поставленої проблеми.

Введені критерії, які повною мірою оцінюють найважливіші аспекти проблеми - якість розпізнавання руки та помилки розпізнавання.