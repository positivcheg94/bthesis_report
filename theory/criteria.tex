\section{Критерії оцінки якості роботи системи}

Для оцінки адекватності роботи алгоритмів запропоновано ввести такі 2 критерії:
\begin{enumerate}
	\item Відсоток пікселів, що належать руці та були помічені алгоритмом як ті, що належать руці;
	\item Відсоток пікселів, які не належать руці, проте були помічені алгоритмом як ті, що належать.
\end{enumerate}

Цих критеріїв буде достатньо для порівняння алгоритмів.

Перевірка алгоритмів буде проводитись в найкращих для кожного алгоритма умовах:
\begin{enumerate}
	\item Віднімання фону - рука рухаеться весь час на відео, сторонніх об'єктів, що рухаються, немає;
	\item Байесовський класифікатор - при тих самих умовах освітлення що і проводилось навчання та з умовою, що лице не знаходиться у полі зору камери ( оскільки можна провести детекцію лиця на відео і потім вилучити його з досліджуваних областей;
	\item Камера глибини - у полі зору на відстані до 0.5 метрів окрім руки інших об'єктів немає.
\end{enumerate}