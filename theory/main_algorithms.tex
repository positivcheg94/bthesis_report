\section{Основні алгоритми}
\subsection{Віднімання фону}
Віднімання фону дуже поширений алгоритм, який використовується у ситуаціях, коли відомо, що камера відносно нерухома. Завдяки йому можна достатньо просто виділити фон із зображення, а значить і передній план чи просто об'єкти, які динамічно рухаються на відео.

Часто використовується для обробки відео з дорожніх відеорегістраторів чи IP камер, які розташовані у людних місцях. Відніманням фону можна миттево отримати усі рухомі об'єкти на відео - машини, пішоходи тощо. Також є певні модифікації цього алгоритму для виділення навіть об'єктів, що не рухаються певний час, проте вони більш пристосовані під окремі задачі.

\subsubsection{Математичні основи}
\begin{equation}
\label{eq:background_substraction}
P[F(t)] = P[I(t)] - P[B]
\end{equation}
де $P[B]$ - зображення фону, $P[I(t)]$ - поточне зображення, $P[F(t)]$ - різниця між поточним зображенням та фоном.

За умови незмінності фону можна працювати з формулою (\ref{eq:background_substraction}).

Для відслідковування рухів динамічних об'єктів також використовується рівняння:
\begin{equation}
\label{eq:frames_difference}
\|P[F(t)] - P[F(t+1)]\| > Treshold
\end{equation}
де $Treshold$ - деяка задана порогова величина.

\begin{figure}[H]
	\includegraphics[width=0.9\linewidth]{theory/img/background_substraction}
	\caption{Ілюстрація роботи простого алгоритма віднімання фону}
\end{figure}

Тобто маючи попередньо зображення фону можна отримати передній план зображення просто віднявши від поточного зображення поелементно зображення фону.

Також існують підходи, за яких зберігається не фонове зображення, а історія N останніх зображень. Такі алгоритми називають змішаними.

Нехай у деякий момент $t = t^{*}$ у пам'яті зберігається історія $N$ останніх фреймів відео $P[I(t^{*} - 1)], ... , P[I(t^{*} - N)]$, $\lambda_{1}, ... , \lambda_{N} : \sum_{i = 1}^{N} \lambda_{i} = 1$ - ваги, тоді фонове зображення може бути виражене як:
\begin{equation}
	P[F(t^{*})] = \sum_{ i = 1}^{N} \lambda_{i} * P[I(t^{*} - i)]
	\label{eq:background_approximation}
\end{equation}

Від вибору $\lambda_{i}$ залежить значимість віддалених у часі фреймів у фоновому зображенні.Якщо взяти їх усіх рівними $\frac{1}{N}$, то такий алгоритм називають змішаною моделю середнього значення віднімання фону. 

Виділення переднього фону можна записати як:
\begin{equation}
Foreground[I(t^{*})] = | P[F(t^{*})] - P[I(t^{*})] | > Threshold
\label{eq:foreground_extraction}
\end{equation}
$| P[F(t^{*})] - P[I(t^{*})] | > Threshold$ у формулі \ref{eq:foreground_extraction} слід розуміти як поелементна різниця матриць і поелементне порівняння цієї різниці з певною пороговою величиною. Тобто на виході отримане бінарне зображення - матриця з 0 та 1.

\subsubsection{Особливості реалізації}

У роботі використовується змішана гаусівська модель віднімання фону, що описана у \cite{MOG1} та \cite{MOG2}. Така модель краща за звичайну модель середнього значення через те, що чим більше віддалена від поточного зображення фрейм, тим менша його значимість на данний момент. Також гаусівский розподіл дуже добре себе показує у багатьох ситуаціях. Звісно можна вибрати коефіцієнти згідно з рядом фібоначі, проте гаусівський розподіл показує кращі результати.

\subsection{Байесовський класифікатор}
\subsubsection{Теоретичні засади}

Цей підхід зустрічається у літературі дуже часто і основна формула, що описує модель, на перший погляд незрозуміла:
\begin{equation}
\label{eq:bayesian_classifier}
P(s|c) = \frac{P(c|s) * P(s)}{P(c)}
\end{equation}

Це звичайний запис теореми Байеса для знаходження апостеріорних ймовірностей.

$P(s|c)$ - ймовірність того, що піксель належить до множини кольорів шуканого об'єкта за умови що піксель має колір $c$. $P(s|c)$ в свою чергу має обернене формулювання - ймовірність того, що піксель приймає значення $c$ за умови що він належить до множини кольорів об'єкта.
$P(s)$ - ймовірність того, що піксель належить до множини кольорів об'єкта. $P(c)$ - загальна ймовірність того, що піксель має колір $c$.

\subsubsection{Пристосування до задачі розпізнавання кольорів об'єкта}

З формули \ref{eq:bayesian_classifier} не зовсім зрозуміло як взягалі шукати 3 ймовірності, що присутні у правій частині рівності.

Основне припущення цього підходу - кольори окремих пікселей на рображенні незалежні один від одного. За такого припущення можна сказати, що $P(s)/P(c)$ - деяка загальна константа якою можна знехтувати поставивши її рівною 1, а $P(c|s) = m/n$, де $m$ - кількість пікселів кольору $c$, що належать об'єкту, $n$ - загальна кількість пікселів, що належала об'єкту.

\subsubsection{Корекція моделі}
Приймаючи до уваги, що в задачі розпізнавання людської руки на відео класифікатор використовується лише для 2 із 3 компонент колірного простору, в якому є виділення люмінантної складової в окрему компоненту, то результатом навчання класифікатора слід очікувати деяку матрицю ймовірностей.

Основною ідеєю корекції моделі є те, що навчання могло проводитись не зовсім якісно і тому деякі ймовірнсті можуть бути лише поодинокими шумами. Матрицю ймовірностей можна розцінювати як деяке зображення та його обробку.

Запропоновано такі етапи обробки матриці ймовірностей:
\begin{enumerate}
	\item Видалення поодиноких шумів використавши морфологічні операції ерозії та дилації;
	\item Розмиття ймовірностей використовуючи гаусівске ядро;
	\item Провести декомпозицію моделі на підмоделі за допомоги пошуку blob використавши вбудований в бібліотеку opencv SimpleBlobDetector.
\end{enumerate}

\subsubsection{Удосконалення процесу навчання класифікатора}

\subsection{Обробка відеопотоку камери глибини}