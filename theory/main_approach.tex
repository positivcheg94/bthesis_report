\section{Загальний підхід до задачі розпізнавання людської руки на відео}

Загальний підхід до розв'язання поставленої задачі можна розбити на окремі етапи:
\begin{enumerate}
	\item Отримання поточного зображення з камери - у випадку простої RGB камери це лише отримання поточного зображення, проте у випадку Intel Realsense камери цей етап може потребувати деяких простих обчислень. Оскільки камера від компанії Intel має окрім звичайного RGB сенсору ще й інфрачервоний для отримання карти глибини може виникнути проблема синхронізації двох відеопотоків оскільки фізично ці два сенсори знаходяться на деякій відстані один від одного. Методи для синхронізації це звичайні алгоритми синхронізації відеопотоків 2 чи більше камер зі стереоскопії. Саме такий і реалізований у драйвері librealsense.
	\item Попередня обробка зображення - використовується для стабілізації будь-якого цифрового зображеня припускаючи що сенсори не ідеальні та можуть вносити деяку похибку, яку можливо зменшити використавши фільтри зглажування: box фильтр, гаусівський фільтр чи фильтр медіаною.
	\item Застосування вибраного алгоритму до зображення - цей етап також можна узагальнити для трьох вибраних підходів оскільки в усіх випадках вхідні дані алгоритму складаються лише з зображення, а вихідні з бінарного зображення з виділеними областями в яких рнймовірно локалізована людська рука.
	\item Заключна обробка вихідного результату вибраного алгоритму - отримавши бінарне зображення з виділеними областями потрібно обробити усі області та відфільтрувати ті, що по певним параметрам немає сенсу розглядати. Наприклад, вважаючи, що рука на зображенні повинна мати розмір більший ніж деяка фіксована величина можливо відсіяти велику кількість малих за площею областей. Те саме стосується і занадто великих областей, які за деяких причин були помічені алгоритмом як рука, проте вони завеликі для обробки і їх також немає сенсу розглядати.
\end{enumerate}

На виході буде отримано бінарне зображення з областями, в яких ймовірно знаходиться людська. На цьому задача локалізації людської руки завершується і цей результат може подаватися на вхід алгоритмів по аналізу жестів. Які працюють з кожною областю окремо та можуть проводити свою спеціальну перевірку на те, чи знаходиться у цій області людська рука.