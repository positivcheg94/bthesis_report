\likechapter{Висновки по роботі та перспективи подальших досліджень}

У ході роботи були реалізовани 3 кардинально різні підходи:
\begin{enumerate}
	\item Віднімання фону;
	\item Байесовський класифікатор;
	\item Аналіз depth відеопотоку.
\end{enumerate}

Віднімання фону та використання камери глибини дали дуже точні результати по обом критеріям, проте у загальному випадку віднімання фону саме в задачі локалізації руки на відео використовувати недоцільно через прив'язку до фону та рухомих об'єктів. Обробка відеопотоку камери глибини також дає можливість точно виділити людську руку, проте постають проблеми ціни таких камер та синхронізації depth та кольорового відеопотоків.

Реалізація байесовського класифікатора дозволяє зменшити кількість обчислень та використовувати його на веб камерах бюджетних ноутбуків.