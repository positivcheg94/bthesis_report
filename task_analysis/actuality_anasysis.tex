\section{Aктуальність поставленої задачі}

Розпізнавання людської руки на відео - задача актуальна саме зараз оскільки це основа для більш складного процесу розрізнавання статичних та динамічних жестів. 

Розпізнавання жестів насамперед це лише ідея по роботі з жестами і базується на абсолютно різних технологіях: переносимі пристрої ( рукавиці чи інші контролери), зафіксовані камери ( RGB камери, depth камери чи комбінація RGB камер для отримання стереоскопічного зображення ) чи навіть радари ( працють по принципу сканування магнітного фону та його зміни у часі ).

Застосування можуть мати найрізноманітніші форми:
\begin{itemize}
	\item Керування ПК - у 2016 році вийшли перші ноутбуки з вбудованою камерою Intel Realsense F200 ( замість звичної RGB камери). На данний момент на її базі створюються програмні продукти по керуванню ПК без миші.
	\item Навчання мові німих - аналізуючи жести можна повністю оцифрувати мові німих та створити програму-вчителя, що буде показувати жести і перевіряти наскільки правильно користувач їх повторює
	\item Керування віртуальними середовищами - в кінці 2015 року компанія Microsoft презентувала прототип Hololens, що створений саме для реалізації доповненої реальності та за допомоги камер Kinect аналізувати жести користувача для більш звичної взаємодії з віртуальним середовищем оскільки майже 90\% фізичної взаємодії з середовищем людина виконує за допомоги рук, а тому слід вважати, що це основний спосіб взаємодії. 
\end{itemize}

Причини складності задачі:
\begin{itemize}
	\item Нестабільні умови освітленності - освітленність дуже сильно впливає на алгоритми, що базуються на обробці RGB відеопотоку;
	\item Поява сторонніх об'єктів у полі зору камери - дуже сильно впливає оскільки збільшує ймовірність розпізнати цей об'єкт як той, що потребує подальшого аналізу;
	\item Недостатня якість камер призводить до того, що в зображенні присутні шуми;
	\item Прості RGB камери мають неякісні матриці і через це реєструють кольори спотворено ускладнює роботу з колірними ознаками шкіри людини;
	\item Колір шкіри людей різниться і тому навчання класифікаторів потрібно проводити на достатньо великих вибірках.	 
\end{itemize}

