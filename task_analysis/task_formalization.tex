\section{Формалізація постановки задачі дослідження}
\jointitles
\subsection{Формальний опис задачі}

Вхідними данними є відеопотік з кольорової камери чи камери глибини. Потрібно реалізувати алгоритми обробки кожного фрейма відеопотоку та алгоритми локалізації людської руки.

Основними алгоритмами локалізації руки на відео вибрані такі:
\begin{enumerate}
	\item Віднімання фону;
	\item Байесовський класифікатор;
	\item Обробка відеопотоку камери глибини.
\end{enumerate}

Також потрібно ввести критерії для оцінки роботи алгоритмів та можливості подальшого їх порівняння.

Останнім етапом роботи є порівняння алгоритмів з урахуванням таких факторів: складність обчислень та швидкість обробки, оптимальні умови роботи і вартість обладнання.

Основні цілі:
\begin{enumerate}
	\item Реалізувати класичний та найпоширеніший метод, що використовується для вирішення поставленої задачі - віднімання фону;
	\item Реалізувати та удосконалити байесовський класифікатор для класифікації кольору людської шкіри на відео і локалізації руки. Удосконалити метод навчання класифікатора;
	\item Представити метод обробки відеопотоку камери глибини як найбільш сучасний підхід до вирішення поставленої задачі.
\end{enumerate}
