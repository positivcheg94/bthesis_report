% !TeX spellcheck = uk_UA
\likechapternotoc{РЕФЕРАТ}

Дипломна робота: \pageref*{MyLastPage}~ст. , \totfig~рис.,  \tottab~табл., \total{citenum}~джерел та 2 додатки.

Метою даної роботи є реалізація методів по локалізації людської руки у відеопотоці. У роботі реалізовані такі методи як: віднімання фону, колірний фільтр на основі байесовскього класифікатора та обробка відеопотоку з камери глибини.

Результати роботи:
\begin{itemize}
	\item реалізовані три підходи по локалізації людської руки на відео;
	\item проведений аналіз умов для надійної роботи алгоритмів;
	\item запропоновано і реалізовано ефективний метод навчання байесовскього класифікатора;
	\item проведено порівняння алгоритмів.
\end{itemize}

Результати даної роботи рекомендовано використовувати у системах взаємодії людини та комп'ютера.
При подальших дослідженнях у цій області доцільно створити систему по взаємодії с комп'ютером на основі реалізованого методу детекції руки на відео.

%Ключові слова:
\MakeUppercase{Розпізнавання об'єктів, взаємодія комп'ютера та людини, камера глибини, Байесовський класифікатор, колірний простір.} 