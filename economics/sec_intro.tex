
У даному розділі проводиться оцінка основних характеристик програмного продукту, призначеного для обробки вхідного відеопотоку і локалізації людської руки. Інтерфейс користувача був розроблений за допомогою мови програмування С++ та opencv.

Програмний продукт призначено для використання на персональних комп’ютерах під управлінням операційних систем Windows або Linux.

Нижче наведено аналіз різних варіантів реалізації модулю з метою вибору оптимальної, з огляду при цьому як на економічні фактори, так і на характеристики продукту, що впливають на продуктивність роботи і на його сумісність з апаратним забезпеченням. Для цього було використано апарат функціонально-вартісного аналізу.

Функціонально-вартісний аналіз (ФВА) -- це технологія, яка дозволяє оцінити реальну вартість продукту або послуги незалежно від організаційної структури компанії. Як прямі, так і побічні витрати розподіляються по продуктам та послугам у залежності від потрібних на кожному етапі виробництва обсягів ресурсів. Виконані на цих етапах дії у контексті метода ФВА називаються функціями.

Мета ФВА полягає у забезпеченні правильного розподілу ресурсів, виділених на виробництво продукції або надання послуг, на прямі та непрямі витрати. У даному випадку -- аналізу функцій програмного продукту й виявлення усіх витрат на реалізацію цих функцій. 

Фактично, цей метод працює за таким алгоритмом:
\begin{itemize}
	\item визначається послідовність функцій, необхідних для виробництва продукту. Спочатку -- всі можливі, потім вони розподіляються по двом групам: ті, що впливають на вартість продукту і ті, що не впливають. На цьому ж етапі оптимізується сама послідовність скороченням кроків, що не впливають на цінність і відповідно витрат.
	\item для кожної функції визначаються повні річні витрати й кількість робочих часів.
	\item для кожної функції на основі оцінок попереднього пункту визначається кількісна характеристика джерел витрат.
	\item після того, як для кожної функції будуть визначені їх джерела витрат, проводиться кінцевий розрахунок витрат на виробництво продукту.
\end{itemize}